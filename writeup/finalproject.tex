\documentclass[10pt]{article}

\usepackage[margin=1in]{geometry}
\usepackage{fancyhdr}
\usepackage{setspace}
\usepackage{lscape}

\usepackage{amsthm}
\usepackage{amsmath}

\usepackage{arydshln}

\usepackage{enumitem}
\setlist{nolistsep}

\usepackage{multirow}
\usepackage{wrapfig}

\usepackage{tikz}
\usetikzlibrary{arrows,automata,shapes.geometric}

\pagestyle{fancy}
\fancyhead[L]{Jessica Huynh}
\fancyhead[C]{Statistical Approaches to Natural Language Processing}
\fancyhead[R]{Final project}
\fancyfoot[C]{\thepage}

\title{Statistical morpheme segmentation in Inuktitut}
\author{Jessica Huynh}
\date{\today}

\begin{document}
	
\maketitle

\onehalfspacing

\section{Introduction}
Inuktitut is an Inuit language spoken mostly in Canada, and like many other Eskimo-Aleut languages, it is highly agglutinative, although not very fusional.\cite{syllabics}

\section{Background}
The highly regular nature of Inuktitut morphology means that one can write a rule-based morpheme segmenter with relatively high accuracy. Indeed, the National Research Council of Canada has already built one, claiming over 95\% accuracy on the Nunavut Hansard corpus and similar accuracy on Inuktitut web pages, composing a list of thousands of roots and suffixes to do so.\cite{analyzer}

\section{Experiment}
Morfessor\cite{morfessor} is a package of 

\subsection{Method}
\cite{morfessor}

\subsection{Approaches}

\subsection{Corpora}
For the corpora, I used the Nunavut Hansard corpus\cite{hansard}

\subsection{Tools}

\section{Results}

\section{Discussion}

\section{Conclusion}

\pagebreak
\bibliography{bib}
\bibliographystyle{plain}

\end{document}